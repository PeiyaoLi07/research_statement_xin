\documentclass{article}
\usepackage[utf8]{inputenc}
\usepackage{fancyhdr}
\usepackage{natbib}
\usepackage{graphicx}
\pagestyle{fancy}
\fancyhf{}
\rhead{}
\lhead{\includegraphics[width=5 cm]{haas_logo.png}}
\rfoot{Page \thepage}
\begin{document}
\title{Xin Chen Research Statement}
\maketitle
\thispagestyle{fancy}
\noindent
\raggedright
My research interest is in understanding marketing empirics revolving around e-commerce and social marketing. My research brings together social science theory, statistical tools, and novel proprietary field data from online markets. I see myself as a questions-driven researcher. My goal is to provide new insights into consumer behavior and firm strategies and to develop new methods as needed. My research addresses questions of both practical relevance and theoretical interests. I devise novel empirical strategies as necessary and apply cutting-edge causal inference tools, such as generalized causal forests. \\
\vspace{1 cm}
\centering
\large
\textbf{Job Market Paper}\\
\normalsize
\noindent
\raggedright
My job market paper examines the effects of bundling social incentives with promotions. When using social promotions, a firm must decide whether or not to require customers to share the promotion on social media with friends. On the one hand, this requirement may make it easier for the firm to get the word out. On the other hand, this requirement may inconvenience customers and lower their propensity to purchase. To conduct valid inference on this trade-off between the costs and benefits, I devise novel empirical strategies that flexibly accommodate the amount of information available to a firm. Through a field experiment on a Chinese online grocery site, randomized both at the city and individual level, I find that social promotions can serve as an effective growth strategy. The social obligation can benefit a firm through two channels. First, the customers who do share add additional value through both new customer acquisition and existing customer retention. Second, the firm does not lose profit from the customers who choose to not purchase the promoted products when forced to share (but would otherwise purchase the promoted products without mandatory sharing); they contribute instead by purchasing other substitutes from the same site. Furthermore, exploiting customer heterogeneity and targeting mandatory sharing to the proper types of customers can further improve the efficacy of social promotions. \\
\vspace{1 cm}
\centering
\large
\textbf{Works in Progress}\\
\normalsize
\noindent
\raggedright
My current works in progress continue on the theme of social marketing and e-commerce. In a joint work with Yunhao Huang titled “Understanding Peer Effects Heterogeneity through Cultural Differences: How Individualism/Collectivism Affects Consumer Sharing Behavior”, we examine how cultural differences affect peer effects in social promotions. Social promotions may create peer effects in that for a customer who is forced to share promotional information with friends on social media, her propensity to share may depend on the fraction of peers who are also required to share. In particular, we hypothesize that compared with individualistic customers, collectivistic customers' propensity to share are more sensitive to the fraction of peers required to share. We test this through an online field experiment that randomizes both at the market and individual level. Based on the "Rice Theory" \citep{RiceTheory} in literature, we construct measures capturing customers' degree of collectivism/individualism through their rice/wheat consumption behavior on the same site. We find that among the relatively individualistic customers who are forced to share,  the average propensity to share does not depend on whether a firm forces a random half or all customers in the same market to share. In contrast, among the relatively collectivistic customers, the average propensity to share increases by 25\% when a firm forces all rather than a random half of customers in the market to share. This implies that when running social promotions in collectivistic cultures, firms should be especially attentive to potential peer effects. Causally evaluating the efficacy of social promotions may require relaxation of the Stable Unit Treatment Value Assumption which is typically assumed in an experimentation context.
\noindent
\raggedright
In another joint work with Yunhao Huang and Liying Qiu titled “Do Social Promotions Turn less Social Capital into Economic Capital over Time? - Evidence from Online Field Experiments," we examines how the efficacy of bundling promotions with social obligations dynamically evolves over time. Through an online field experiment that involves over more than 21,000 customers over 9 months, we find that requiring customers to share promotional information with friends generates fewer referrals over time. On average, requiring sharing generates 5 times the number of referrals (also for referrals who are new and existing customers, respectively) over the first time period (four months) compared with over the second period (six months). The number of ``effective" referrals, i.e. customers who buy within 24 hours of referred site visit, in the first period is 10 times the number in the second period. Underlying the diminishing marginal return on the leads generated, we find suggestive evidence of several mechanisms. First, total site traffic also declines over time and this may be due to more fierce market competition. Compared with in the first period, the daily average number of unique customers visiting the APP decreases by 18\% in the second period. Second, customers' propensity to share decreases by almost 20 percentage points from 47\% to 28\%. Within-customer change accounts for the majority of this decline because the average propensity to share among the entire cohort in the first period decreases from 47\% to 32\%. Third, among the customers who choose to share, a smaller fraction broadcast (on news feed) as opposed to narrow-cast (via direct messaging). Conditional on sharing, the fraction of broadcast referrers decreases significantly from 87\% to 66\%. Using propensity score matching, we find that compared with customers who narrow-cast, those  who broadcast bring in significantly more leads (an average incremental effect of .055 units). Based on our estimates, had the narrow-cast customers decided to broadcast instead, the number of referrals would more than triple. Fourth, the types of customers who self-select into considering the social promotion also change over time. Compared with the first period, the customers who consider the social promotion in the second period are more price sensitive, have a higher probability of having consumed the promoted products before, made their recent purchases longer ago, used gift cards more frequently, and have longer tenure on the site. 
\noindent
\raggedright
In another joint work with Zachary Zhong titled “Bundle Premiums: Evidence from Taobao," we document a surprising empirical regularity in the largest Chinese online market. Unlike what conventional bundle pricing theory suggests, we find that on the largest Chinese e-commerce platform Taobao.com, one needs to pay more for a bundle compared with purchasing each bundle item from the same online market separately. This phenomenon of bundle premium permeates multiple product categories, including digital cameras, iPads, cell-phones, and video game consoles. Conservatively speaking, in the digital single-lens reflex camera market, ``premiumed bundles'' account for 30\% of the market share. Furthermore, the magnitude of the premium is also beyond that of rational expectation. For example, the average premium for the Canon 700D bundles is  17 USD, or 4\% the price of the primary base good.  This bundle premium phenomenon may have significant consumer welfare implications. This paper systematically documents the empirical regularities of these premiums.
\noindent
\raggedright
My other early-stage research revolves around  efficient design of dynamic experimentation and how customer heterogeneity dynamically evolves over time. For example, I study when using social promotions, how a firm should personalize both pricing and the social obligation. 
\noindent
\raggedright
To summarize, my research contributions include bringing together rich novel data to provide insights into individual-level consumer behavior and inform firms’ decision making. Methodologically, my research tools include both field experimentation and machine learning. 
\newpage
\bibliographystyle{plainnat}
\bibliography{reference.bib}
\end{document}



